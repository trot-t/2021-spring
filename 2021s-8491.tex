\providecommand{\pdfxopts}{a-1b,cyrxmp}
%\providecommand{\pdfxopts}{a-1b}
\providecommand{\thisyear}{2021}
\immediate\write18{rm \jobname.xmpdata}%  uncomment for Unix-based systems
\begin{filecontents*}{\jobname.xmpdata}
\Title{Ведомость Датчики, группа 8491, весна \textemdash\thisyear}
\Author{Прокшин Артем Николаевич}
\Creator{pdfTeX + pdfx.sty with options \pdfxopts }
\Subject{Ведомость посещения занятий и выполнения лабораторных работ по датчикам студентами 8491 группы}
\Keywords{ведомость посещения, группа 8491, ЛЭТИ}
\CoverDisplayDate{март \thisyear}
\CoverDate{2021-04-22}
\Copyrighted{True}
\Copyright{Public Domain}
\CopyrightURL{http://github.com/trot-t}
\Creator{pdfTeX + pdfx.sty with options \pdfxopts }
\end{filecontents*}


\documentclass[a4paper,landscape,11pt]{article}

\pdfcompresslevel=9

\usepackage[\pdfxopts]{pdfx}[2016/03/09]
\PassOptionsToPackage{obeyspaces}{url}
\let\tldocrussian=1  % for live4ht.cfg


%\documentclass[a4paper,11pt]{article} 
%\usepackage[T1,T2A]{fontenc}
\usepackage[utf8x]{inputenc}
\usepackage[english,russian]{babel} 
\usepackage{wrapfig}
%\usepackage[table,xcdraw]{xcolor}
\usepackage{booktabs}
\usepackage{pifont}
\usepackage{graphicx}
\graphicspath{ {images/} }

\usepackage{tikz}
\usepackage{siunitx}
\usepackage[american,cuteinductors,smartlabels]{circuitikz}

\usepackage{hyperref}

\usepackage{advdate}
%\usepackage{showframe} % для отладки позиции на странице
\usepackage{cancel}

%\setlength{\voffset}{-72pt} %отступ сверху - чтобы увидеть откомментарить \usepackage{showframe}
\setlength{\voffset}{-56pt} %landscape
\setlength{\topmargin}{0pt} 
%\setlength{\headheight{1pt}
\setlength{\headsep}{0pt}
\setlength{\hoffset}{-222pt} %landscape
\setlength{\marginparwidth}{0pt}
\setlength{\textwidth}{800pt} %landscape
\setlength{\textheight}{538pt} %landscape
\setlength{\footskip}{-60pt}


%\author{ Прокшин Артем \\
%\small ЛЭТИ\\
%\small \texttt{taybola@gmail.com}}

%\date{}
%abcdefghijklmnop


\newcommand*\OK{&\small \ding{51}$\!\!_\circ$} % начал защищать
\newcommand*\Ok{&\small \ding{51}$\!\!_\circ$} % начал защищатi
\newcommand*\ok{&{\small \ding{51}}} % присутствовал
\newcommand*\oK{&{\small \ding{51}?}} % присутствовал?
\newcommand*\no{&{\small }} % отсутствовал
\newcommand*\D{\tiny\ding{48}} % защита, defend
\newcommand*\da{&{\small\ding{48}$\!\!_1$}} % защита, defend
\newcommand*\dab{&{\small\ding{48}$\!\!^1_2$}} % защита, defend
\newcommand*\ab{&{\small\ding{48}$\!\!^1_2$}} % защита, defend
\newcommand*\ad{&{\small${}^1\!\!$\ding{48}$\!\!_4$}} % защита, defend
%\newcommand*\ab{&{\small\ding{48}$\!\!^1_2$}} % защита, defend
\newcommand*\bc{&{\small\ding{48}$\!\!^2_3$}} % защита, defend
\newcommand*\dabc{&{\small\ding{48}$\!\!^1_{23}$}} % защита, defend
\newcommand*\dabcd{&{\small\ding{48}$\!\!^{12}_{34}$}} % защита, defend
\newcommand*\ac{&{\small\ding{48}$\!\!^1_{23}$}} % защита, defend
\newcommand*\db{&{\small\ding{48}$\!\!_2$}} % защита, defend
\newcommand*\dc{&{\small\ding{48}$\!\!_3$}} % защита, defend
\newcommand*\dd{&{\small\ding{48}$\!\!_4$}} % защита, defend
\newcommand*\bd{&{\small${}^2\!\!$\ding{48}$\!\!^3_{4}$}} % защита, defend
\newcommand*\de{&{\small\ding{48}$\!\!_5$}} % защита, defend
\newcommand*\dE{&{\small${}^4\!\!\!$\ding{48}$\!\!_5$}} % защита, defend
\newcommand*\cd{&{\small\ding{48}$\!\!^3_4$}} % защита, defend
\newcommand*\dg{&{\small\ding{48}$\!\!_6$}} % защита, defend
\newcommand*\fg{&{\small${}^6\!\!$\ding{48}$\!\!_7$}} % защита, defend
\newcommand*\dH{&{\small\ding{48}$\!\!_8$}} % защита, defend
\newcommand*\gh{&{\small\ding{48}$\!\!^7_8$}} % защита, defend
\newcommand*\fh{&{\small\ding{48}$\!\!^7_{89}$}} % защита, defend 
\newcommand*\ce{&{\small${}^3\!\!$\ding{48}$\!\!_5$}} % защита, defend
\newcommand*\ef{&{\small${}^5\!\!$\ding{48}$\!\!_6$}} % защита, defend
%\newcommand*\dh{&{\small\ding{48}$\!\!_8$}} % защита, defend
\newcommand*\di{&{\small\ding{48}$\!\!_9$}} % защита, defend
\newcommand*\cdef{&{\small ${}^2_4\!\!$\ding{48}$\!\!^{3}_{5}$}} % защита, defend
\newcommand*\cde{&{\small ${}^2\!\!$\ding{48}$\!\!^{3}_{5}$}} % защита, defend
\newcommand*\efg{&{\small ${}^5\!\!$\ding{48}$\!\!^{6}_{7}$}} % защита, defend
\newcommand*\befgh{&{\small ${}_2^5\!\!$\ding{48}$\!\!^{6}_{78}$}} % защита, defend
\newcommand*\Dh{&{\small${}^4\!\!$\ding{48}$\!\!_8$}} % защита, defend
\newcommand*\cfg{&{\small ${}^3\!\!$\ding{48}$\!\!^{6}_{7}$}} % защита, defend
\newcommand*\fgh{&{\small ${}^6\!\!$\ding{48}$\!\!^{7}_{8}$}} % защита, defend
\newcommand*\bce{&{\small ${}^2\!\!$\ding{48}$\!\!^{3}_{5}$}} % защита, defend
\newcommand*\dO{&{\small\ding{48}$\!\!_{15}$}}
\newcommand*\Skip{\noindent\rule{0.3cm}{0.9pt}}


\begin{document}
%\thispagestyle{empty}
% or
\pagenumbering{gobble}
%\AdvanceDate[-1] % печатаю в субботу а нужна пятница
\begin{center}\today\end{center}
\vspace*{1\baselineskip} %landscape

%\begin{table} \centering 
\newcommand*{\CS}{9pt} % ширина колонки
\newcommand*{\CT}{15pt} % ширина колонки с температурой
\begin{tabular}{p{7pt}|l|p{\CS}|p{\CS}|p{\CT}|p{\CS}|p{\CS}|p{\CS}|p{\CS}|p{\CS}|p{\CS}}
%\multicolumn{16}{c}{График выполнения лабораторных работ студентами 8871 группы} \\ 
\multicolumn{11}{c}{Ведомость посещения занятий по датчикам студентами 8491 группы} \\
\toprule 
&&&&&&&&&&\\
&&&&&&&&&&\\
&&&&&&&&&&\\
&&&&&&&&&&\\
&&&&&&&&&&\\
&&&&&&&&&&\\
&&&&&&&&&&\\
&&&&&&&&&&\\
&&\rotatebox{90}{\rlap{\small 6 марта ( ОУ )}}
& \rotatebox{90}{\rlap{\small 20 марта /инстр.У)}}
& \rotatebox{90}{\rlap{\small 3 апреля /фильтр}}
& \rotatebox{90}{\rlap{\small /напряж.сети}}
& \rotatebox{90}{\rlap{\small 15 мая /синхрониз.}}
& \rotatebox{90}{\rlap{\small 29 мая}}
& \rotatebox{90}{\rlap{\small  }}
& \rotatebox{90}{\rlap{\small }}
& \rotatebox{90}{\rlap{\small }}
\\
% commands vi to copy/paste D :+19 ->>> p :-18 :w
\midrule
1\,&   Аврамёнок Дмитрий Андреевич        \ok\no\ok  \ok\ok\ok&&\\
2\,&   Андреев Степан Дмитриевич          \ok\no\no  \no\no\no&&\\
3\,&   Галкин Павел Дмитриевич            \ok\no\ok  \ok\ok\ok&&\\
4\,&   Герасимов Егор Сергеевич           \ok\no&35.9\ok\ok\no&&\\
5\,&   Игнатович Юлия Васильевна          \ok\ok&34.7\ok\ok\ok&&\\
\midrule
6\,&   Иевлев Максим Денисович            \ok\no\no  \no\no\no&&\\
7\,&   Илёсов Комрон Гиёс угли            \no\no\no  \no\no\no&&\\
8\,&   Карамашин Илья Александрович       \no\no\no  \ok\no\no&&\\
9\,&   Каримжонов Хусниддин Темуржон угли \no\no\no  \ok\ok\no&&\\
10\,&  Келлер Елизавета Александровна     \ok\ok\ok  \ok\ok\ok&\\
\midrule
11\,&  Кузьмин Игорь Леонидович           \ok\no\no  \ok\ok\ok&&\\ 
12\,&  Купецкова Марина Борисовна         \ok\no\no  \ok\no\ok&&\\
13\,&  Курасбеков Ердос                   \ok\no\no  \ok\no\ok&&\\
14\,&  Масленникова Екатерина Андреевна   \ok\ok&35.3\ok\ok\ok&\\
15\,&  Назаренко Анастасия Александровна  \ok\no\no  \no\no\no&&\\
\midrule
16\,&  Пеньщиков Вадим Вадимович          \no\no&35.7\no\no\no&&\\
17\,&  Пшебельская Регина Сергеевна       \ok\no\no  \ok\no\no&&\\ 
18\,&  Пышногуб Артем Александрович       \ok\no\no  \no\ok\ok&&\\
19\,&  Рыбаков Иннокентий Викторович      \no\no\no  \no\no\no&&\\
20\,&  Саламахин Антон                    \ok\no&35.7\ok\ok\ok&&\\
\midrule
21\,&  Смирнов Иван Алексеевич            \no\no\no  \ok\ok\no&&\\
22\,&  Смирнов Никита Валерьевич          \no\no\no  \no\no\no&&\\
23\,&  Туленков Кирилл Алексеевич         \ok\no\no  \no\no\no&&\\
\bottomrule
\end{tabular} 

\newpage
%
\begin{tabular}{l|llccccccccccccc}
\multicolumn{10}{c}{выполнение лабораторнах работ по датчикам, 8491 группа} \\
\toprule
&&Л1&Л1& Л2&Л2& Л3&Л3& Л4&Л4 &Л5&Л5& Л6&Л6\\
\midrule
1\,&   Аврамёнок Дмитрий Андреевич        &17.04&17.04& 14.05 & 15.05& 14.05& 15.05& 15.05& 15.05& 27.05& 29.05\\
2\,&   Андреев Степан Дмитриевич                  &19.03& 7.06& 21.05& 5.06 &&&&\\
3\,&   Галкин Павел Дмитриевич            & 12.03& 20.03& 5.05& 29.05& 6.05& 29.05& 20.05 & 29.05& 20.05& 29.05\\
4\,&   Герасимов Егор Сергеевич           & 16.03& 27.03& 29.03& 15.05& 17.04& 17.04 & 8.05& 15.05 & 8.05& 15.05\\
5\,&   Игнатович Юлия Васильевна          & 16.03& 20.03& 1.04& 3.04& 17.04& 15.05& 15.05& 15.05 & 24.05 & 29.05\\
\midrule
6\,&   Иевлев Максим Денисович            &21.03& 5.06&&&&&&&&&14.05&29.05\\
7\,&   Илёсов Комрон Гиёс угли            &&&&&&&&\\
8\,&   Карамашин Илья Александрович       &20.04& 5.06&&&&&&\\
9\,&   Каримжонов Хусниддин Темуржон угли &27.05& 5.06& 26.05& 5.06&&&&\\
10\,&  Келлер Елизавета Александровна     & 30.04& 15.05 &30.04& 15.05 & 7.05& 15.05&16.05& 29.05& 28.05 & 29.05\\
\midrule
11\,&  Кузьмин Игорь Леонидович           &15.05 & 15.05& 15.05& 15.05& 22.05& 29.05& 22.05& 29.05 & 29.05 & 29.05\\
12\,&  Купецкова Марина Борисовна         &21.05& 29.05& &&&&&\\
13\,&  Курасбеков Ердос                   &28.05& 5.06&&&&&&\\
14\,&  Масленникова Екатерина Андреевна   & 16.03& 20.03&30.03& 3.04& 17.04& 15.05& 14.05& 15.05 & 18.05& 29.05\\
15\,&  Назаренко Анастасия Александровна  & 3.06& 5.06&&&&&&\\
\midrule
16\,&  Пеньщиков Вадим Вадимович          &&&&&&&&\\
17\,&  Пшебельская Регина Сергеевна       &&&&&&&&\\
18\,&  Пышногуб Артем Александрович       &27.05 & 5.06&&&&&&\\
19\,&  Рыбаков Иннокентий Викторович      &20.04& 5.06& 12.05& 5.06&&&&\\
20\,&  Саламахин Антон                    & 2.04& 3.04& 17.04& 15.05& 14.05& 15.05& 14.05& 15.05 & 26.05 & 29.05\\
\midrule
21\,&  Смирнов Иван Алексеевич            &28.03&17.05& 2.04& 15.05 &16.04 & 17.04& 9.05& 15.05 & 9.05& 15.05\\
22\,&  Смирнов Никита Валерьевич          &&&&&&&&\\
23\,&  Туленков Кирилл Алексеевич         &1.06& 5.06&&&&&&\\
\bottomrule
\end{tabular}

\subsection*{пр.1 Операционный усилитель}
\begin{itemize}
	\item Галкин -- нет таблицы $E_1,E_2$. Файлы с исходными должны быть в виде 849103\_01\_02.txt (01-я лабораторная, 02-е задание в лабораторной).
в последнем задании нужно было номинал всех  резисторов увеличить на 100 ($K_u$ при этом не меняется). у автора увеличены ТОЛЬКО резисторы обратной связи  1.3MEG

\item Игнатович -- в экпортированном файле резисторы обратной связи не соответствуют заданию
\includegraphics[scale=0.2]{"849105_01_01"}
\begin{verbatim}
R4          2 3 10K
\end{verbatim}
\item Герасимов -- %п.9 $E_\text{смещения}3.21 мВ и  $E_\text{смещения}2.96 мВ -- какой верный?
имена файлов c моделямми в формате pspice поменять с laba12.txt на 849104\_01.txt и т.д.

\item Андреев имя файла отчета для лаб.1 должно быть 849102\_01.pdf, имена файлов схем 849102\_01\_01.txt
\item   Смирнов Иван в файле отчета  849121\_01\_2.txt несимметрия R1=1k,R3=30.1Meg и R2=1k,R4=31k привела к тому что в 1000 раз увеличилось $U_\text{вх}$  при $E_1=0, E_2=0$
\end{itemize}

\subsection*{Фильтры}
\begin{itemize}
\item Герасимов --  заменить на модель рельного операционного усилителя
\item Смирнов --  заменить на модель рельного операционного усилителя
\end{itemize}
\end{document}
