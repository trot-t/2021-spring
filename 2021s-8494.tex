\providecommand{\pdfxopts}{a-1b,cyrxmp}
%\providecommand{\pdfxopts}{a-1b}
\providecommand{\thisyear}{2021}
\immediate\write18{rm \jobname.xmpdata}%  uncomment for Unix-based systems
\begin{filecontents*}{\jobname.xmpdata}
\Title{Ведомость Датчики, группа 8494, весна\textemdash\thisyear} % год поставится сам
\Author{Прокшин Артем Николаевич}
\Creator{pdfTeX + pdfx.sty with options \pdfxopts }
\Subject{Ведомость посещения занятий и выполнения лабораторных работ по датчикам студентами 8494 группы}
\Keywords{ведомость посещения, группа 8494, ЛЭТИ}
\CoverDisplayDate{март \thisyear}
\CoverDate{2021-04-22}
\Copyrighted{True}
\Copyright{Public Domain}
\CopyrightURL{http://github.com/trot-t}
\Creator{pdfTeX + pdfx.sty with options \pdfxopts }
\end{filecontents*}


\documentclass[a4paper,landscape,11pt]{article}

\pdfcompresslevel=9

\usepackage[\pdfxopts]{pdfx}[2016/03/09]
\PassOptionsToPackage{obeyspaces}{url}
\let\tldocrussian=1  % for live4ht.cfg

%\documentclass[a4paper,11pt]{article} 
%\usepackage[T1,T2A]{fontenc}
\usepackage[utf8x]{inputenc}
\usepackage[english,russian]{babel} 
\usepackage{wrapfig}
%\usepackage[table,xcdraw]{xcolor}
\usepackage{booktabs}
\usepackage{pifont}
\usepackage{graphicx}
\graphicspath{ {images/} }

\usepackage{tikz}
\usepackage{siunitx}
\usepackage[american,cuteinductors,smartlabels]{circuitikz}

\usepackage{hyperref}

\usepackage{advdate}
%\usepackage{showframe} % для отладки позиции на странице
\usepackage{cancel}

%\setlength{\voffset}{-72pt} %отступ сверху - чтобы увидеть откомментарить \usepackage{showframe}
\setlength{\voffset}{-56pt} %landscape
\setlength{\topmargin}{0pt} 
%\setlength{\headheight{1pt}
\setlength{\headsep}{0pt}
\setlength{\hoffset}{-222pt} %landscape
\setlength{\marginparwidth}{0pt}
\setlength{\textwidth}{800pt} %landscape
\setlength{\textheight}{538pt} %landscape
\setlength{\footskip}{-60pt}


%\author{ Прокшин Артем \\
%\small ЛЭТИ\\
%\small \texttt{taybola@gmail.com}}

%\date{}
%abcdefghijklmnop
\newcommand*\OK{&\small \ding{51}$\!\!_\circ$} % начал защищать
\newcommand*\Ok{&\small \ding{51}$\!\!_\circ$} % начал защищатi
\newcommand*\ok{&{\small \ding{51}}} % присутствовал
\newcommand*\oK{&{\small \ding{51}?}} % присутствовал?
\newcommand*\no{&{\small }} % отсутствовал
\newcommand*\D{\tiny\ding{48}} % защита, defend
\newcommand*\da{&{\small\ding{48}$\!\!_1$}} % защита, defend
\newcommand*\dab{&{\small\ding{48}$\!\!^1_2$}} % защита, defend
\newcommand*\ab{&{\small\ding{48}$\!\!^1_2$}} % защита, defend
\newcommand*\ad{&{\small${}^1\!\!$\ding{48}$\!\!_4$}} % защита, defend
%\newcommand*\ab{&{\small\ding{48}$\!\!^1_2$}} % защита, defend
\newcommand*\bc{&{\small\ding{48}$\!\!^2_3$}} % защита, defend
\newcommand*\dabc{&{\small\ding{48}$\!\!^1_{23}$}} % защита, defend
\newcommand*\dabcd{&{\small\ding{48}$\!\!^{12}_{34}$}} % защита, defend
\newcommand*\ac{&{\small\ding{48}$\!\!^1_{23}$}} % защита, defend
\newcommand*\db{&{\small\ding{48}$\!\!_2$}} % защита, defend
\newcommand*\dc{&{\small\ding{48}$\!\!_3$}} % защита, defend
\newcommand*\dd{&{\small\ding{48}$\!\!_4$}} % защита, defend
\newcommand*\bd{&{\small${}^2\!\!$\ding{48}$\!\!^3_{4}$}} % защита, defend
\newcommand*\de{&{\small\ding{48}$\!\!_5$}} % защита, defend
\newcommand*\dE{&{\small${}^4\!\!\!$\ding{48}$\!\!_5$}} % защита, defend
\newcommand*\cd{&{\small\ding{48}$\!\!^3_4$}} % защита, defend
\newcommand*\dg{&{\small\ding{48}$\!\!_6$}} % защита, defend
\newcommand*\fg{&{\small${}^6\!\!$\ding{48}$\!\!_7$}} % защита, defend
\newcommand*\dH{&{\small\ding{48}$\!\!_8$}} % защита, defend
\newcommand*\gh{&{\small\ding{48}$\!\!^7_8$}} % защита, defend
\newcommand*\fh{&{\small\ding{48}$\!\!^7_{89}$}} % защита, defend 
\newcommand*\ce{&{\small${}^3\!\!$\ding{48}$\!\!_5$}} % защита, defend
\newcommand*\ef{&{\small${}^5\!\!$\ding{48}$\!\!_6$}} % защита, defend
%\newcommand*\dh{&{\small\ding{48}$\!\!_8$}} % защита, defend
\newcommand*\di{&{\small\ding{48}$\!\!_9$}} % защита, defend
\newcommand*\cdef{&{\small ${}^2_4\!\!$\ding{48}$\!\!^{3}_{5}$}} % защита, defend
\newcommand*\cde{&{\small ${}^2\!\!$\ding{48}$\!\!^{3}_{5}$}} % защита, defend
\newcommand*\efg{&{\small ${}^5\!\!$\ding{48}$\!\!^{6}_{7}$}} % защита, defend
\newcommand*\befgh{&{\small ${}_2^5\!\!$\ding{48}$\!\!^{6}_{78}$}} % защита, defend
\newcommand*\Dh{&{\small${}^4\!\!$\ding{48}$\!\!_8$}} % защита, defend
\newcommand*\cfg{&{\small ${}^3\!\!$\ding{48}$\!\!^{6}_{7}$}} % защита, defend
\newcommand*\fgh{&{\small ${}^6\!\!$\ding{48}$\!\!^{7}_{8}$}} % защита, defend
\newcommand*\bce{&{\small ${}^2\!\!$\ding{48}$\!\!^{3}_{5}$}} % защита, defend
\newcommand*\dO{&{\small\ding{48}$\!\!_{15}$}}
\newcommand*\Skip{\noindent\rule{0.3cm}{0.9pt}}


\begin{document}
%\thispagestyle{empty}
% or
\pagenumbering{gobble}
%\AdvanceDate[-1] % печатаю в субботу а нужна пятница
\begin{center}\today\end{center}
\vspace*{1\baselineskip} %landscape

%\begin{table} \centering 
\newcommand*{\CS}{9pt} % ширина колонки
\begin{tabular}{p{7pt}|l|p{\CS}|p{\CS}|p{\CS}|p{\CS}|p{\CS}|p{\CS}|p{\CS}|p{\CS}|p{\CS}}
%\multicolumn{16}{c}{График выполнения лабораторных работ студентами 8871 группы} \\ 
\multicolumn{11}{c}{Ведомость посещения занятий по датчикам студентами 8494 группы} \\
\toprule 
&&&&&&&&&&\\
&&&&&&&&&&\\
&&&&&&&&&&\\
&&&&&&&&&&\\
&&&&&&&&&&\\
&&&&&&&&&&\\
&&&&&&&&&&\\
&&&&&&&&&&\\
&&\rotatebox{90}{\rlap{\small 13 февраля ( ОУ )}}
 &\rotatebox{90}{\rlap{\small 27 февраля /инстр.OУ}}
 &\rotatebox{90}{\rlap{\small 13 марта/избират.фильтр}}
 &\rotatebox{90}{\rlap{\small 27 марта/напряж.сети}}
 &\rotatebox{90}{\rlap{\small 10 апреля/синхрониз.}}
 &\rotatebox{90}{\rlap{\small 24 апреля/датчики ТАД}}
 &\rotatebox{90}{\rlap{\small 22 мая }}
 &\rotatebox{90}{\rlap{\small }}
 &\rotatebox{90}{\rlap{\small }}
\\
% commands vi to copy/paste D :+19 ->>> p :-18 :w
\midrule
1\,&   Анохин Петр Александрович       \ok\ok\ok\ok\ok\ok\no&\\
2\,&   Врионакис Анна Андреевна        \ok\ok\ok\ok\ok\ok\ok&\\
3\,&   Гавришенков Андрей Андреевич    \ok\ok\no\no\no\ok\ok&\\
4\,&   Голиков Дмитрий Александрович   \ok\ok\ok\ok\ok\ok\ok&\\
5\,&   Дамизов Георгий Игоревич        \ok\no\ok\no\no\no\ok&\\
\midrule
6\,&   Егоров Денис Валерьевич         \ok\no\ok\ok\no\ok\no&\\
7\,&   Кондаурова Анна Евгеньевна      \ok\no\no\ok\no\no\no&\\
8\,&   Кораблев Иван                   \no\no\no\no\no\no\ok&\\
9\,&   Кусаинов Асылхан                \ok\ok\ok\ok\ok\ok\no&\\ 
10\,&  Мазуров Глеб Алексеевич         \ok\no\ok\no\ok\ok\no&\\
\midrule
11\,&  Машкалев Никита Александрович   \ok\no\ok\no\no\no\no&\\
12\,&  Парфёнов Антон Русланович       \ok\no\ok\ok\no\no\no&\\
13\,&  Слепков Виктор Владимирович     \ok\ok\no\no\no\no\no&\\
14\,&  Сосновская Анастасия Алексеевна \ok\ok\ok\ok\ok\ok\no&\\
\bottomrule
\end{tabular} 

\newpage
%
\begin{tabular}{l|llccccccccccccc}
\multicolumn{10}{c}{выполнение лабораторнах работ по датчикам, 8494 группа} \\
\toprule
&&Л1&Л1& Л2&Л2& Л3&Л3& Л4&Л4& Л5&Л5& Л6&Л6\\
\midrule
1\,&   Анохин Петр Александрович       &3.05& 5.05&&&&&&&&&5.06& 5.06\\
2\,&   Врионакис Анна Андреевна        &14.03&27.03&22.05&22.05&22.05&22.05&22.05&22.05& 22.05&22.05& 5.06& 5.06\\
3\,&   Гавришенков Андрей Андреевич    &5.06& 5.06&&&&&&&&& 5.06& 5.06\\
4\,&   Голиков Дмитрий Александрович   &26.04&22.05& 4.05 &22.05& 4.05 &22.05&14.05&22.05&14.05&22.05& 4.06& 5.06\\
5\,&   Дамизов Георгий Игоревич        & 5.06& 5.06&&&&&&\\
\midrule
6\,&   Егоров Денис Валерьевич         &31.03&22.05& --    & 22.05 &&&&\\
7\,&   Кондаурова Анна Евгеньевна      &&&&&&&&\\
8\,&   Кораблев Иван                   &2.06& 5.06&&&&&&\\
9\,&   Кусаинов Асылхан                &31.05& 5.06&&&&&&\\
10\,&  Мазуров Глеб Алексеевич         &&&&&&&&\\
\midrule
11\,&  Машкалев Никита Александрович   &27.05& 5.06&&&&&&&&& 4.06& 5.06\\
12\,&  Парфёнов Антон Русланович       &5.06 & 5.06&&&&&&\\
13\,&  Слепков Виктор Владимирович     &&&&&&&&\\
	14\,&  Сосновская Анастасия Алексеевна & 2.06& 5.06&&&&&&&&& 4.06 & 5.06\\
\bottomrule
\end{tabular}

\subsection*{пр.1}
\begin{itemize}
	\item Врионакис -- в свойствах pdf-файле отсутствуют ключевые слова и автор. $K_u$ для измерения $E_\text{смещения}$ следовало бы задать больше. (849402\_01\_08\_09.txt)
	\item Пятовский --
		где графики? надо вставить графики.
почему выход операционного усилителя соединен с землей.
пожалуйста прикрепите к отчету в moodle файл   OPA277.cir
\end{itemize}

\subsection{л2}
\begin{itemize}
\item Егоров -- оба напряжения питания ОУ должны быть 14Вольт
\item Врионакис -- почему-то при 10m для VG1 и VG2 на выходе напряжение 10В
\item Голиков -- если на входе системы 5мВ, а на выходе 6мВ, то коэффициент усиления $K_{U\text{синфазное}} = 1.2$, а не $6^10^-3$ 
\end{itemize}
\end{document}
