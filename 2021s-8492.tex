\documentclass[a4paper,landscape,11pt]{article}
%\documentclass[a4paper,11pt]{article} 
%\usepackage[T1,T2A]{fontenc}
\usepackage[utf8x]{inputenc}
\usepackage[english,russian]{babel} 
\usepackage{wrapfig}
\usepackage[table,xcdraw]{xcolor}
\usepackage{booktabs}
\usepackage{pifont}
\usepackage{graphicx}
\graphicspath{ {images/} }

\usepackage{tikz}
\usepackage{siunitx}
\usepackage[american,cuteinductors,smartlabels]{circuitikz}

\usepackage{hyperref}

\usepackage{advdate}
%\usepackage{showframe} % для отладки позиции на странице
\usepackage{cancel}

%\setlength{\voffset}{-72pt} %отступ сверху - чтобы увидеть откомментарить \usepackage{showframe}
\setlength{\voffset}{-56pt} %landscape
\setlength{\topmargin}{0pt} 
%\setlength{\headheight{1pt}
\setlength{\headsep}{0pt}
\setlength{\hoffset}{-222pt} %landscape
\setlength{\marginparwidth}{0pt}
\setlength{\textwidth}{800pt} %landscape
\setlength{\textheight}{538pt} %landscape
\setlength{\footskip}{-60pt}


%\author{ Прокшин Артем \\
%\small ЛЭТИ\\
%\small \texttt{taybola@gmail.com}}

%\date{}
%abcdefghijklmnop


\newcommand*\OK{&\small \ding{51}$\!\!_\circ$} % начал защищать
\newcommand*\Ok{&\small \ding{51}$\!\!_\circ$} % начал защищатi
\newcommand*\ok{&{\small \ding{51}}} % присутствовал
\newcommand*\oK{&{\small \ding{51}?}} % присутствовал?
\newcommand*\no{&{\small }} % отсутствовал
\newcommand*\D{\tiny\ding{48}} % защита, defend
\newcommand*\da{&{\small\ding{48}$\!\!_1$}} % защита, defend
\newcommand*\dab{&{\small\ding{48}$\!\!^1_2$}} % защита, defend
\newcommand*\ab{&{\small\ding{48}$\!\!^1_2$}} % защита, defend
\newcommand*\ad{&{\small${}^1\!\!$\ding{48}$\!\!_4$}} % защита, defend
%\newcommand*\ab{&{\small\ding{48}$\!\!^1_2$}} % защита, defend
\newcommand*\bc{&{\small\ding{48}$\!\!^2_3$}} % защита, defend
\newcommand*\dabc{&{\small\ding{48}$\!\!^1_{23}$}} % защита, defend
\newcommand*\dabcd{&{\small\ding{48}$\!\!^{12}_{34}$}} % защита, defend
\newcommand*\ac{&{\small\ding{48}$\!\!^1_{23}$}} % защита, defend
\newcommand*\db{&{\small\ding{48}$\!\!_2$}} % защита, defend
\newcommand*\dc{&{\small\ding{48}$\!\!_3$}} % защита, defend
\newcommand*\dd{&{\small\ding{48}$\!\!_4$}} % защита, defend
\newcommand*\bd{&{\small${}^2\!\!$\ding{48}$\!\!^3_{4}$}} % защита, defend
\newcommand*\de{&{\small\ding{48}$\!\!_5$}} % защита, defend
\newcommand*\dE{&{\small${}^4\!\!\!$\ding{48}$\!\!_5$}} % защита, defend
\newcommand*\cd{&{\small\ding{48}$\!\!^3_4$}} % защита, defend
\newcommand*\dg{&{\small\ding{48}$\!\!_6$}} % защита, defend
\newcommand*\fg{&{\small${}^6\!\!$\ding{48}$\!\!_7$}} % защита, defend
\newcommand*\dH{&{\small\ding{48}$\!\!_8$}} % защита, defend
\newcommand*\gh{&{\small\ding{48}$\!\!^7_8$}} % защита, defend
\newcommand*\fh{&{\small\ding{48}$\!\!^7_{89}$}} % защита, defend 
\newcommand*\ce{&{\small${}^3\!\!$\ding{48}$\!\!_5$}} % защита, defend
\newcommand*\ef{&{\small${}^5\!\!$\ding{48}$\!\!_6$}} % защита, defend
%\newcommand*\dh{&{\small\ding{48}$\!\!_8$}} % защита, defend
\newcommand*\di{&{\small\ding{48}$\!\!_9$}} % защита, defend
\newcommand*\cdef{&{\small ${}^2_4\!\!$\ding{48}$\!\!^{3}_{5}$}} % защита, defend
\newcommand*\cde{&{\small ${}^2\!\!$\ding{48}$\!\!^{3}_{5}$}} % защита, defend
\newcommand*\efg{&{\small ${}^5\!\!$\ding{48}$\!\!^{6}_{7}$}} % защита, defend
\newcommand*\befgh{&{\small ${}_2^5\!\!$\ding{48}$\!\!^{6}_{78}$}} % защита, defend
\newcommand*\Dh{&{\small${}^4\!\!$\ding{48}$\!\!_8$}} % защита, defend
\newcommand*\cfg{&{\small ${}^3\!\!$\ding{48}$\!\!^{6}_{7}$}} % защита, defend
\newcommand*\fgh{&{\small ${}^6\!\!$\ding{48}$\!\!^{7}_{8}$}} % защита, defend
\newcommand*\bce{&{\small ${}^2\!\!$\ding{48}$\!\!^{3}_{5}$}} % защита, defend
\newcommand*\dO{&{\small\ding{48}$\!\!_{15}$}}
\newcommand*\Skip{\noindent\rule{0.3cm}{0.9pt}}


\begin{document}
%\thispagestyle{empty}
% or
\pagenumbering{gobble}
%\AdvanceDate[-1] % печатаю в субботу а нужна пятница
\begin{center}\today\end{center}
\vspace*{1\baselineskip} %landscape

%\begin{table} \centering 
\newcommand*{\CS}{9pt} % ширина колонки
\begin{tabular}{p{7pt}|l|p{\CS}|p{\CS}|p{\CS}|p{\CS}|p{\CS}|p{\CS}|p{\CS}|p{\CS}|p{\CS}}
%\multicolumn{16}{c}{График выполнения лабораторных работ студентами 8871 группы} \\ 
\multicolumn{11}{c}{Ведомость посещения занятий по датчикам студентами 8492 группы} \\
\toprule 
&&&&&&&&&&\\
&&&&&&&&&&\\
&&&&&&&&&&\\
&&&&&&&&&&\\
&&&&&&&&&&\\
&&&&&&&&&&\\
&&&&&&&&&&\\
&&&&&&&&&&\\
&&\rotatebox{90}{\rlap{\small 13 февраля ( ОУ )}}
 &\rotatebox{90}{\rlap{\small 27 февраля /инстр.OУ}}
 &\rotatebox{90}{\rlap{\small 13 марта/избират.фильтр}}
 &\rotatebox{90}{\rlap{\small 27 марта/напряж.сети}}
 &\rotatebox{90}{\rlap{\small 10 апреля/синхрониз.}}
 &\rotatebox{90}{\rlap{\small 24 апреля/датчики ТАД}}
 &\rotatebox{90}{\rlap{\small 22 мая }}
 &\rotatebox{90}{\rlap{\small }}
 &\rotatebox{90}{\rlap{\small }}
\\
% commands vi to copy/paste D :+19 ->>> p :-18 :w
\midrule
1\,&  Анискин Максим Николаевич        \ok\ok\ok\ok\ok&&&\\
2\,&  Басан Константин Андреевич       \ok\ok\ok\ok\ok&&&\\
3\,&  Василевская Алёна Александровна  \ok\ok\ok\no\ok&&&\\
4\,&  Волобуева Яна Сергеевна          \ok\ok\ok\ok\ok&&&\\
5\,&  Вольвачёва Анна Валерьевна       \ok\ok\ok\no\ok&&&\\
\midrule
6\,&  Драгунов Артур Александрович     \ok\ok\o\no\ok&&&&\\
7\,&  Зайцев Михаил Евгеньевич         \no\no\no\no\no&&&\\
8\,&  Занин Никита Сергеевич           \ok\ok\no\ok\ok&&&\\
9\,&  Кирпичёнок Дарья Сергеевна       \ok\no\ok\ok\ok&&&\\ 
10\,& Киселёва Дарья Алексеевна        \ok\ok\ok\ok\no&&&\\
\midrule
11\,& Клычков Владислав Максимович     \ok\no\ok\ok\ok&&&\\
12\,& Крестников Евгений Александрович \ok\ok\ok\ok\no&&&\\
13\,& Кушлевец Злата Денисовна         \ok\ok\ok\no\ok&&&\\
14\,& Лебедева Ксения Николаевна       \ok\ok\ok\ok\ok&&&\\
15\,& Литвяков Иван Леонидович         \ok\ok\ok\ok\ok&&&\\
\midrule
16\,& Лыс{о}в Александр Сергеевич        \ok\ok\ok\no\ok&&&\\
17\,& Мансуров Артем Тимурович         \o\ok\ok\no\ok&&&\\
18\,& Нерсесов Артем Михайлович        \ok\ok\ok\ok\ok&&&\\ 
19\,& Огаркова Полина Игоревна         \ok\ok\ok\ok\ok&&&\\
20\,& Пятовский Максим Андреевич       \ok\ok\ok\no\ok&&&\\
\midrule
21\,& Рихсиев Шухратжон Рахимжон угли  \ok\ok\ok\no\no&&&\\
22\,& Рыженков Алексей Михайлович      \ok\ok\ok\no\ok&&&\\
23\,& Чирков Владислав Сергеевич       \ok\ok\ok\ok\ok&&&\\
\bottomrule
\end{tabular} 

\newpage
%
\begin{tabular}{l|llccccccccccccc}
\multicolumn{10}{c}{выполнение лабораторнах работ по датчикам, 8492 группа} \\
\toprule
&&Л1&Л1& Л2&Л2& Л3&Л3& Л4&Л4& &Л5&Л5& Л6&Л6\\
\midrule
1\,&  Анискин Максим Николаевич        &&&&&&&&\\
2\,&  Басан Константин Андреевич       & 12.06& -- &&&&&&\\
3\,&  Василевская Алёна Александровна  & 12.03& -- &&&&&&\\
4\,&  Волобуева Яна Сергеевна          & 15.03& 27.03&&&&&&\\
5\,&  Вольвачёва Анна Валерьевна       & 13.03& -- &&&&&&\\
\midrule
6\,&  Драгунов Артур Александрович     & 12.03& 27.03&&&&&&\\
7\,&  Зайцев Михаил Евгеньевич         &&&&&&&&\\
8\,&  Занин Никита Сергеевич           &&&&&&&&\\
9\,&  Кирпичёнок Дарья Сергеевна       & 26.03& 27.03&&&&&&\\
10\,& Киселёва Дарья Алексеевна        & 26.03& 27.03&&&&&&\\
\midrule
11\,& Клычков Владислав Максимович     & 23.03& 27.03&&&&&&\\
12\,& Крестников Евгений Александрович &&&&&&&&\\
13\,& Кушлевец Злата Денисовна         &&&&&&&&\\
14\,& Лебедева Ксения Николаевна       & 17.03& 27.03 &&&&&&\\
15\,& Литвяков Иван Леонидович         &&&&&&&&\\
\midrule
16\,& Лысов Александр Сергеевич        & 12.03& 26.03&&&&&&\\
17\,& Мансуров Артем Тимурович         &&&&&&&&\\
18\,& Нерсесов Артем Михайлович        &&&&&&&&\\
19\,& Огаркова Полина Игоревна         & 12.03& --  &&&&&&\\
20\,& Пятовский Максим Андреевич       & 23.03& 27.03&&&&&&\\
\midrule
21\,& Рихсиев Шухратжон Рахимжон угли  &&&&&&&&\\
22\,& Рыженков Алексей Михайлович      &&&&&&&&\\
23\,& Чирков Владислав Сергеевич       &&&&&&&&\\
\bottomrule
\end{tabular}

\subsection*{пр. № 1}
\begin{itemize}
\item
	общая ошибка -- питение ОУ обычно $\pm 5V, \pm 15V$ или  $\pm 25V$ и никакого отношения к коэфициенту усиления по напряжению не имеет 
\item
Басан почему автор pdf-файла Влад Володин? УГО сопротивлений не по ГОСТ. если бы в отчёте был график $U_\text{вых}(U_\text{вх})$ то стало бы видно что питание операционных усилителей 2В вместо 25В.
приведенная в отчете схема в части питания операционного усилителя неверна
\begin{verbatim}
VS3 4 5 30
\end{verbatim}

\begin{circuitikz}
\draw node[op amp] (opamp) {}
(opamp.up) --++ (2,0) to[battery] ++(0,-1.1) -| (opamp.down);
	\draw[red,thick] (1.5,-0.5) -- (2.5,0.5)
;\end{circuitikz}
		при E1=E2=0 в вашем отчете 182 uV, в 2.42393e-05 - это напряжение $U_\text{вых}$ при $K_u=12$ при проверке.

питание ОУ выбрано 2В а не 15В
\begin{verbatim}
VS4         0 1 2
VS3         2 0 2
\end{verbatim}

\item 
	Василевская -- почему автор pdf-файла Влад Володин?  УГО сопротивлений не по ГОСТ. нет исходных файлов в формате pspice с расширением txt. 
	если бы в отчёте был график $U_\text{вых}(U_\text{вх})$ то стало бы видно что питание операционных усилителей 2В вместо 25В. приведенная в отчете схема в части питания операционного усилителя неверна.
	см. отзывы на работу Басан.

\item
        Вольвачева -- почему автор pdf-файла Влад Володин? УГО сопротивлений не по ГОСТ. нет исходных файлов в формате pspice с расширением txt.
        если бы в отчёте был график $U_\text{вых}(U_\text{вх})$ то стало бы видно что питание операционных усилителей 600mВ вместо 25В. 
	см. отзывы на работу Басан.

\item
	Огаркова -- питание ОУ выбрано 0в  в схеме, должно быть подано питание ОУ
\item
	Киселева Д.А. -- питание ОУ выбрано нестандартное 10в  в схеме Обычно бывает $\pm 15В$ (или $\pm25В$) , поэтому получить $U_\text{выходное}$ больше чем 10 вольт не могли.
		Однако таблица приведена для идеального ОУ. Вставить проверяющего в основную надпись. В шапке должна быть литера У - учебный.
		для текстовых файлов использовать нумерацию 849210\_01\_01.txt
\item  
	Драгунов А.А. -- для напряжения смещения добавить $K_u$.

\item 
	Кирпичёнок рис. 2.1 не ваш вариант, пожалуйста добавьие грвфики  $U_\text{выходное}(U_\text{входное})$ для реального ОУ при инвертирующем и неинвертирующем включении
	питание ОУ выбрано 9В.  Обычно бывает $\pm 15В$ (или $\pm25В$) 
\begin{verbatim}
VS4         4 0 9
VS3         0 5 9
\end{verbatim}
имена  исходных файлов в формате pspice с расширением txt gприводите в формате 849201\_01\_01.txt

\item 
	Волобуева -- напряжение смещения 3.21 мВ или 2.1 мВ? 
напряжение питания, попали в распросстраненный стандарт $\pm5$ B
\begin{verbatim}
VS4         1 0 5
VS3         0 2 5
\end{verbatim}

\item 
	Лебедева -- похоже, что стала образцом для работы Кирпичёнок, по крайней мере в рис 2.1

\end{itemize}

\subsection*{Лабораторная 2}
\begin{itemize}
	\item Басан -- в ПЭ3  УГО не по ГОСТу (в принципиальных схемах)
по каким измерениям получены магические
величины $U_\text{дифф}$ и $U_\text{син}$
\item Василевская -- непонятно как измерено
$K_\text{подавления синфазного сигнала}$
\end{itemize}

\end{document}
