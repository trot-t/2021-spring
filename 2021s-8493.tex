\documentclass[a4paper,landscape,11pt]{article}
%\documentclass[a4paper,11pt]{article} 
%\usepackage[T1,T2A]{fontenc}
\usepackage[utf8x]{inputenc}
\usepackage[english,russian]{babel} 
\usepackage{wrapfig}
\usepackage[table,xcdraw]{xcolor}
\usepackage{booktabs}
\usepackage{pifont}
\usepackage{graphicx}
\graphicspath{ {images/} }

\usepackage{tikz}
\usepackage{siunitx}
\usepackage[american,cuteinductors,smartlabels]{circuitikz}

\usepackage{hyperref}

\usepackage{advdate}
%\usepackage{showframe} % для отладки позиции на странице
\usepackage{cancel}

%\setlength{\voffset}{-72pt} %отступ сверху - чтобы увидеть откомментарить \usepackage{showframe}
\setlength{\voffset}{-56pt} %landscape
\setlength{\topmargin}{0pt} 
%\setlength{\headheight{1pt}
\setlength{\headsep}{0pt}
\setlength{\hoffset}{-222pt} %landscape
\setlength{\marginparwidth}{0pt}
\setlength{\textwidth}{800pt} %landscape
\setlength{\textheight}{538pt} %landscape
\setlength{\footskip}{-60pt}


%\author{ Прокшин Артем \\
%\small ЛЭТИ\\
%\small \texttt{taybola@gmail.com}}

%\date{}
%abcdefghijklmnop


\newcommand*\OK{&\small \ding{51}$\!\!_\circ$} % начал защищать
\newcommand*\Ok{&\small \ding{51}$\!\!_\circ$} % начал защищатi
\newcommand*\ok{&{\small \ding{51}}} % присутствовал
\newcommand*\oK{&{\small \ding{51}?}} % присутствовал?
\newcommand*\no{&{\small }} % отсутствовал
\newcommand*\D{\tiny\ding{48}} % защита, defend
\newcommand*\da{&{\small\ding{48}$\!\!_1$}} % защита, defend
\newcommand*\dab{&{\small\ding{48}$\!\!^1_2$}} % защита, defend
\newcommand*\ab{&{\small\ding{48}$\!\!^1_2$}} % защита, defend
\newcommand*\ad{&{\small${}^1\!\!$\ding{48}$\!\!_4$}} % защита, defend
%\newcommand*\ab{&{\small\ding{48}$\!\!^1_2$}} % защита, defend
\newcommand*\bc{&{\small\ding{48}$\!\!^2_3$}} % защита, defend
\newcommand*\dabc{&{\small\ding{48}$\!\!^1_{23}$}} % защита, defend
\newcommand*\dabcd{&{\small\ding{48}$\!\!^{12}_{34}$}} % защита, defend
\newcommand*\ac{&{\small\ding{48}$\!\!^1_{23}$}} % защита, defend
\newcommand*\db{&{\small\ding{48}$\!\!_2$}} % защита, defend
\newcommand*\dc{&{\small\ding{48}$\!\!_3$}} % защита, defend
\newcommand*\dd{&{\small\ding{48}$\!\!_4$}} % защита, defend
\newcommand*\bd{&{\small${}^2\!\!$\ding{48}$\!\!^3_{4}$}} % защита, defend
\newcommand*\de{&{\small\ding{48}$\!\!_5$}} % защита, defend
\newcommand*\dE{&{\small${}^4\!\!\!$\ding{48}$\!\!_5$}} % защита, defend
\newcommand*\cd{&{\small\ding{48}$\!\!^3_4$}} % защита, defend
\newcommand*\dg{&{\small\ding{48}$\!\!_6$}} % защита, defend
\newcommand*\fg{&{\small${}^6\!\!$\ding{48}$\!\!_7$}} % защита, defend
\newcommand*\dH{&{\small\ding{48}$\!\!_8$}} % защита, defend
\newcommand*\gh{&{\small\ding{48}$\!\!^7_8$}} % защита, defend
\newcommand*\fh{&{\small\ding{48}$\!\!^7_{89}$}} % защита, defend 
\newcommand*\ce{&{\small${}^3\!\!$\ding{48}$\!\!_5$}} % защита, defend
\newcommand*\ef{&{\small${}^5\!\!$\ding{48}$\!\!_6$}} % защита, defend
%\newcommand*\dh{&{\small\ding{48}$\!\!_8$}} % защита, defend
\newcommand*\di{&{\small\ding{48}$\!\!_9$}} % защита, defend
\newcommand*\cdef{&{\small ${}^2_4\!\!$\ding{48}$\!\!^{3}_{5}$}} % защита, defend
\newcommand*\cde{&{\small ${}^2\!\!$\ding{48}$\!\!^{3}_{5}$}} % защита, defend
\newcommand*\efg{&{\small ${}^5\!\!$\ding{48}$\!\!^{6}_{7}$}} % защита, defend
\newcommand*\befgh{&{\small ${}_2^5\!\!$\ding{48}$\!\!^{6}_{78}$}} % защита, defend
\newcommand*\Dh{&{\small${}^4\!\!$\ding{48}$\!\!_8$}} % защита, defend
\newcommand*\cfg{&{\small ${}^3\!\!$\ding{48}$\!\!^{6}_{7}$}} % защита, defend
\newcommand*\fgh{&{\small ${}^6\!\!$\ding{48}$\!\!^{7}_{8}$}} % защита, defend
\newcommand*\bce{&{\small ${}^2\!\!$\ding{48}$\!\!^{3}_{5}$}} % защита, defend
\newcommand*\dO{&{\small\ding{48}$\!\!_{15}$}}
\newcommand*\Skip{\noindent\rule{0.3cm}{0.9pt}}


\begin{document}
%\thispagestyle{empty}
% or
\pagenumbering{gobble}
%\AdvanceDate[-1] % печатаю в субботу а нужна пятница
\begin{center}\today\end{center}
\vspace*{1\baselineskip} %landscape

%\begin{table} \centering 
\newcommand*{\CS}{9pt} % ширина колонки
\newcommand*{\CT}{15pt} % ширина колонки с температурой
	\begin{tabular}{p{7pt}|l|p{\CS}|p{\CS}|p{\CT}|p{\CS}|p{\CS}|p{\CS}|p{\CS}|p{\CS}|p{\CS}}
%\multicolumn{16}{c}{График выполнения лабораторных работ студентами 8871 группы} \\ 
\multicolumn{11}{c}{Ведомость посещения занятий по датчикам студентами 8493 группы} \\
\toprule 
&&&&&&&&&&\\
&&&&&&&&&&\\
&&&&&&&&&&\\
&&&&&&&&&&\\
&&&&&&&&&&\\
&&&&&&&&&&\\
&&\rotatebox{90}{\rlap{\small 6 марта ( ОУ )}}
&\rotatebox{90}{\rlap{\small 20 марта (инстр.У)}}
&\rotatebox{90}{\rlap{\small 3 яапреля (фильтр)}}
&\rotatebox{90}{\rlap{\small }}
&\rotatebox{90}{\rlap{\small }}
&\rotatebox{90}{\rlap{\small }}
&\rotatebox{90}{\rlap{\small }}
&\rotatebox{90}{\rlap{\small }}
&\rotatebox{90}{\rlap{\small }}
\\
% commands vi to copy/paste D :+19 ->>> p :-18 :w
\midrule
1\,&  Андросов Глеб Вячеславович      \ok\no\no&&&&&\\
2\,&  Баханов Сергей Евгеньевич       \ok\ok\no&&&&&\\
3\,&  Богданов Иван Александрович     \ok\ok\no&&&&&\\
4\,&  Василенко Василий Александрович \ok\ok\no&&&&&\\
5\,&  Волвенков Алексей Алексеевич    \no\no\no&&&&&\\
\midrule
	6\,&  Геймонен Эрик Ринатович         \no\no&35.9&&&&&\\
	7\,&  Генсер Иван Владимирович        \ok\ok&36.5&&&&\\
8\,&  Дейнеко Иван Сергеевич          \ok\ok\ok&&&&&\\
9\,&  Жуков Кирилл                    \no\no\no&&&&&\\ 
10\,& Коробицына Юлия Алексеевна      \no\no\no&&&&&\\
\midrule
11\,& Кучумов Александр Андреевич     \ok\no\ok&&&&&\\
	12\,& Милюков Михаил Павлович         \ok\ok&36.6&&&&&\\
	13\,& Панарин Андрей Сергеевич        \ok\ok&36.3&&&&&\\
14\,& Панков Никита Алексеевич        \no\no\no&&&&&\\
15\,& Петрова Юлия Дмитриевна         \ok\ok\no&&&&&\\
\midrule
16\,& Размочаев Руслан Евгеньевич     \ok\ok\no&&&&&\\
17\,& Савин Григорий Вячеславович     \ok\ok\no&&&&&\\
	18\,& Селезнев Владимир Алексеевич    \ok\no&36.2&&&&&\\ 
	19\,& Стожков Константин Антонович    \no\no&36.7&&&&&\\
	20\,& Шкаровский Денис Валерьевич     \no\ok&36.6&&&&&\\
\midrule
21\,& Ярунов Егор Романович           \ok\ok\ok&&&&&\\
\bottomrule
\end{tabular} 

\newpage
%
\begin{tabular}{l|llccccccccccccc}
\multicolumn{10}{c}{выполнение лабораторнах работ по датчикам, 8493 группа} \\
\toprule
&&Л1&Л1& Л2&Л2& Л3&Л3& Л4&Л4& &Л5&Л5& Л6&Л6\\
\midrule
1\,&  Андросов Глеб Вячеславович      &      &      &&&&&&\\
2\,&  Баханов Сергей Евгеньевич       & 22.02& 24.02&&&&&&\\
3\,&  Богданов Иван Александрович     &      &      &&&&&&\\
4\,&  Василенко Василий Александрович &      &      &&&&&&\\
5\,&  Волвенков Алексей Алексеевич    &      &      &&&&&&\\
\midrule                                            
6\,&  Геймонен Эрик Ринатович         &      &      &&&&&&\\
7\,&  Генсер Иван Владимирович        &      &      &&&&&&\\
8\,&  Дейнеко Иван Сергеевич          &      &      &&&&&&\\
9\,&  Жуков Кирилл                    &      &      &&&&&&\\
10\,& Коробицына Юлия Алексеевна      &      &      &&&&&&\\
\midrule                                            
11\,& Кучумов Александр Андреевич     &      &      &&&&&&\\
12\,& Милюков Михаил Павлович         &      &      &&&&&&\\
13\,& Панарин Андрей Сергеевич        &      &      &&&&&&\\
14\,& Панков Никита Алексеевич        &      &      &&&&&&\\
15\,& Петрова Юлия Дмитриевна         &      &      &&&&&&\\
\midrule                                            
16\,& Размочаев Руслан Евгеньевич     & 24.02& 24.02& 27.03&  3.04&&&&\\
17\,& Савин Григорий Вячеславович     & 19.03& 20.03&&&&&&\\
18\,& Селезнев Владимир Алексеевич    & 21.03& 03.04&&&&&&\\
19\,& Стожков Константин Антонович    &      &      &&&&&&\\
20\,& Шкаровский Денис Валерьевич     &      &      &&&&&&\\
\midrule                                            
21\,& Ярунов Егор Романович           &      &      &&&&&&\\
\bottomrule
\end{tabular}

\subsection{работа № 1}
Бахранов  -- представьте, что все пришлют работы с названием файла "датчики\_лаб1.pdf", поменяйте на  849302\_01.pdf. Отсутствует исходный код моделей в формате pspice.
В свойствах файла отсутствует заголовок, тема, от автора присутствует только имя, отсутствуют ключевые слова.

Размочаев -- мне пришлось переименовывать файл(локально), поменяйте, пожалуйста имя файла на  849316\_01.pdf
\begin{verbatim}
mv 'Размочаев 8493 лаб1.pdf' 849316_01.pdf
\end{verbatim}
В свойствах файла отсутствует заголовок, тема, присутствует странный автор "Malov", отсутствуют ключевые слова. Отсутствует исходный код моделей в формате pspice.
Резисторы на принципиальных схемах изображены не по ГОСТу.

Савин -- нет графиков Uвых от Uвх ($K_u$) реального операционного усилителя, в последнем задании нужно было номинал всех  резисторов увеличить на 100 ($K_u$ при этом не меняется) 
у автора увеличены резисторы обратной связи 2.7MEG 

\subsection{работа № 2}
замечание ко всем: УГО резисторов по ГОСТ и убрать грязь(решетка) со схем.

Селезнев -- по анализу формулы $U/|\frac{2R_2 + R_1}{R_1}| \approx U/|\frac{2R_2}{R_1}|$ не видно что с изменением $R_1$
напряжение смещения практически не изменяется, а с увеличением $R_2$
уменьшается и с уменьшением $R_2$ – увеличивается


Шкаровский в исходном коде 849320\_0103 узел (8) резистора обратной связи не соединен с выходом ОУ (3)
\begin{verbatim}
XU1         1 2 4 5 3 OPA277_0
R4          2 8 30K
\end{verbatim}
\end{document}
